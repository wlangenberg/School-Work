% Options for packages loaded elsewhere
\PassOptionsToPackage{unicode}{hyperref}
\PassOptionsToPackage{hyphens}{url}
%
\documentclass[
]{article}
\usepackage{lmodern}
\usepackage{amssymb,amsmath}
\usepackage{ifxetex,ifluatex}
\ifnum 0\ifxetex 1\fi\ifluatex 1\fi=0 % if pdftex
  \usepackage[T1]{fontenc}
  \usepackage[utf8]{inputenc}
  \usepackage{textcomp} % provide euro and other symbols
\else % if luatex or xetex
  \usepackage{unicode-math}
  \defaultfontfeatures{Scale=MatchLowercase}
  \defaultfontfeatures[\rmfamily]{Ligatures=TeX,Scale=1}
\fi
% Use upquote if available, for straight quotes in verbatim environments
\IfFileExists{upquote.sty}{\usepackage{upquote}}{}
\IfFileExists{microtype.sty}{% use microtype if available
  \usepackage[]{microtype}
  \UseMicrotypeSet[protrusion]{basicmath} % disable protrusion for tt fonts
}{}
\makeatletter
\@ifundefined{KOMAClassName}{% if non-KOMA class
  \IfFileExists{parskip.sty}{%
    \usepackage{parskip}
  }{% else
    \setlength{\parindent}{0pt}
    \setlength{\parskip}{6pt plus 2pt minus 1pt}}
}{% if KOMA class
  \KOMAoptions{parskip=half}}
\makeatother
\usepackage{xcolor}
\IfFileExists{xurl.sty}{\usepackage{xurl}}{} % add URL line breaks if available
\IfFileExists{bookmark.sty}{\usepackage{bookmark}}{\usepackage{hyperref}}
\hypersetup{
  pdftitle={Case Study 1},
  pdfauthor={Willie Langenberg},
  hidelinks,
  pdfcreator={LaTeX via pandoc}}
\urlstyle{same} % disable monospaced font for URLs
\usepackage[margin=1in]{geometry}
\usepackage{color}
\usepackage{fancyvrb}
\newcommand{\VerbBar}{|}
\newcommand{\VERB}{\Verb[commandchars=\\\{\}]}
\DefineVerbatimEnvironment{Highlighting}{Verbatim}{commandchars=\\\{\}}
% Add ',fontsize=\small' for more characters per line
\usepackage{framed}
\definecolor{shadecolor}{RGB}{248,248,248}
\newenvironment{Shaded}{\begin{snugshade}}{\end{snugshade}}
\newcommand{\AlertTok}[1]{\textcolor[rgb]{0.94,0.16,0.16}{#1}}
\newcommand{\AnnotationTok}[1]{\textcolor[rgb]{0.56,0.35,0.01}{\textbf{\textit{#1}}}}
\newcommand{\AttributeTok}[1]{\textcolor[rgb]{0.77,0.63,0.00}{#1}}
\newcommand{\BaseNTok}[1]{\textcolor[rgb]{0.00,0.00,0.81}{#1}}
\newcommand{\BuiltInTok}[1]{#1}
\newcommand{\CharTok}[1]{\textcolor[rgb]{0.31,0.60,0.02}{#1}}
\newcommand{\CommentTok}[1]{\textcolor[rgb]{0.56,0.35,0.01}{\textit{#1}}}
\newcommand{\CommentVarTok}[1]{\textcolor[rgb]{0.56,0.35,0.01}{\textbf{\textit{#1}}}}
\newcommand{\ConstantTok}[1]{\textcolor[rgb]{0.00,0.00,0.00}{#1}}
\newcommand{\ControlFlowTok}[1]{\textcolor[rgb]{0.13,0.29,0.53}{\textbf{#1}}}
\newcommand{\DataTypeTok}[1]{\textcolor[rgb]{0.13,0.29,0.53}{#1}}
\newcommand{\DecValTok}[1]{\textcolor[rgb]{0.00,0.00,0.81}{#1}}
\newcommand{\DocumentationTok}[1]{\textcolor[rgb]{0.56,0.35,0.01}{\textbf{\textit{#1}}}}
\newcommand{\ErrorTok}[1]{\textcolor[rgb]{0.64,0.00,0.00}{\textbf{#1}}}
\newcommand{\ExtensionTok}[1]{#1}
\newcommand{\FloatTok}[1]{\textcolor[rgb]{0.00,0.00,0.81}{#1}}
\newcommand{\FunctionTok}[1]{\textcolor[rgb]{0.00,0.00,0.00}{#1}}
\newcommand{\ImportTok}[1]{#1}
\newcommand{\InformationTok}[1]{\textcolor[rgb]{0.56,0.35,0.01}{\textbf{\textit{#1}}}}
\newcommand{\KeywordTok}[1]{\textcolor[rgb]{0.13,0.29,0.53}{\textbf{#1}}}
\newcommand{\NormalTok}[1]{#1}
\newcommand{\OperatorTok}[1]{\textcolor[rgb]{0.81,0.36,0.00}{\textbf{#1}}}
\newcommand{\OtherTok}[1]{\textcolor[rgb]{0.56,0.35,0.01}{#1}}
\newcommand{\PreprocessorTok}[1]{\textcolor[rgb]{0.56,0.35,0.01}{\textit{#1}}}
\newcommand{\RegionMarkerTok}[1]{#1}
\newcommand{\SpecialCharTok}[1]{\textcolor[rgb]{0.00,0.00,0.00}{#1}}
\newcommand{\SpecialStringTok}[1]{\textcolor[rgb]{0.31,0.60,0.02}{#1}}
\newcommand{\StringTok}[1]{\textcolor[rgb]{0.31,0.60,0.02}{#1}}
\newcommand{\VariableTok}[1]{\textcolor[rgb]{0.00,0.00,0.00}{#1}}
\newcommand{\VerbatimStringTok}[1]{\textcolor[rgb]{0.31,0.60,0.02}{#1}}
\newcommand{\WarningTok}[1]{\textcolor[rgb]{0.56,0.35,0.01}{\textbf{\textit{#1}}}}
\usepackage{graphicx,grffile}
\makeatletter
\def\maxwidth{\ifdim\Gin@nat@width>\linewidth\linewidth\else\Gin@nat@width\fi}
\def\maxheight{\ifdim\Gin@nat@height>\textheight\textheight\else\Gin@nat@height\fi}
\makeatother
% Scale images if necessary, so that they will not overflow the page
% margins by default, and it is still possible to overwrite the defaults
% using explicit options in \includegraphics[width, height, ...]{}
\setkeys{Gin}{width=\maxwidth,height=\maxheight,keepaspectratio}
% Set default figure placement to htbp
\makeatletter
\def\fps@figure{htbp}
\makeatother
\setlength{\emergencystretch}{3em} % prevent overfull lines
\providecommand{\tightlist}{%
  \setlength{\itemsep}{0pt}\setlength{\parskip}{0pt}}
\setcounter{secnumdepth}{-\maxdimen} % remove section numbering

\title{Case Study 1}
\author{Willie Langenberg}
\date{2021-01-14}

\begin{document}
\maketitle

\hypertarget{exercise-1}{%
\subsection{Exercise 1}\label{exercise-1}}

We are given data generated by an AR model. We are then supposed to find
the number of lags used in the model, and it's parameters. An important
note is that the lags is not above 5. This is essentially \emph{order
determination}, so to find the number of lags one must calculate the AIC
for models with 1, 2, 3, 4 and 5 number of lags, and choose the one with
lowest AIC.

\begin{Shaded}
\begin{Highlighting}[]
\CommentTok{# Read the data}
\NormalTok{AR_df <-}\StringTok{ }\KeywordTok{scan}\NormalTok{(}\StringTok{"AR.txt"}\NormalTok{)}

\CommentTok{# Fit the data to an AR model with max 5 lags.}
\NormalTok{model <-}\StringTok{ }\KeywordTok{ar}\NormalTok{(AR_df, }\DataTypeTok{AIC=}\OtherTok{TRUE}\NormalTok{, }\DataTypeTok{method=}\StringTok{"mle"}\NormalTok{, }\DataTypeTok{order.max =} \DecValTok{5}\NormalTok{)}

\CommentTok{# The function "ar" automatically fits the best model according to it's AIC. }
\CommentTok{# We now want to see what amount of lags it used in the model.}
\NormalTok{model_ord <-}\StringTok{ }\NormalTok{model}\OperatorTok{$}\NormalTok{order}

\CommentTok{# Parameters of the model}
\NormalTok{coeff <-}\StringTok{ }\NormalTok{model}\OperatorTok{$}\NormalTok{ar}
\end{Highlighting}
\end{Shaded}

The best amount of lags to use according to AIC is then 3. The
parameters in the fitted AR(3) model is -0.1971113, 0.2953276,
0.0130519.

\newpage

\hypertarget{exercise-2}{%
\subsection{Exercise 2}\label{exercise-2}}

Given the data for U.S. quarterly real gross domestic product from 1947
to 2010, we are now going to test the null hypothesis \$H\_0 : \rho\_1 =
\rho\emph{2 = \dotsc = \rho}\{12\} = 0 \$ against the alternative
hypothesis \$H\_1 : \rho\_i \neq 0 \textrm{ for some } i \in \{
1,\dotsc,12 \} \$ . We do this by using the Ljung-Box test. The test
statistic is given by
\[Q = n(n+2)\sum_{k=1}^h \frac{\hat{\rho}_k^2}{n-k},\] where \(n\) is
the number of observations, \(\hat{\rho}\) is the autocorrelation at lag
\(k\), and \(h\) is the number of lags being tested. Under the null
hypotheis, \(Q\) asymptotically follows a \(\chi^2(h)\) distribution.

\begin{verbatim}
## 
##  Box-Ljung test
## 
## data:  growth_rate
## X-squared = 62.848, df = 12, p-value = 6.796e-09
\end{verbatim}

The p-value of the test is very small (\textless0.0001\%), so we can
safely reject the null hypothesis. This suggests that the quarterly
growth rate of the GDP is serially correlated.

\begin{Shaded}
\begin{Highlighting}[]
\CommentTok{# Simple AR model ->}
\NormalTok{model2 <-}\StringTok{ }\KeywordTok{ar}\NormalTok{(growth_rate, }\DataTypeTok{AIC=}\OtherTok{TRUE}\NormalTok{, }\DataTypeTok{method=}\StringTok{"mle"}\NormalTok{, }\DataTypeTok{order.max =} \DecValTok{5}\NormalTok{)}
\NormalTok{model2}
\end{Highlighting}
\end{Shaded}

\begin{verbatim}
## 
## Call:
## ar(x = growth_rate, order.max = 5, method = "mle", AIC = TRUE)
## 
## Coefficients:
##       1        2        3  
##  0.3461   0.1298  -0.1224  
## 
## Order selected 3  sigma^2 estimated as  8.258e-05
\end{verbatim}

\begin{Shaded}
\begin{Highlighting}[]
\CommentTok{# Perform model checking to justify your model ->}
\end{Highlighting}
\end{Shaded}

\hypertarget{section}{%
\subsubsection{}\label{section}}

Interpret the results!

Build a suitable AR model, and perform model checking.

\hypertarget{section-1}{%
\subsubsection{}\label{section-1}}

\hypertarget{exercise-3}{%
\subsection{Exercise 3}\label{exercise-3}}

\hypertarget{exercise-3.1}{%
\subsubsection{Exercise 3.1}\label{exercise-3.1}}

\hypertarget{exercise-3.2}{%
\subsubsection{Exercise 3.2}\label{exercise-3.2}}

\end{document}
